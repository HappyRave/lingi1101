% \documentclass{article}
% \usepackage[utf8]{inputenc}
% \usepackage{framed}
% \usepackage{array}
% 
% 
% \title{Logique (P. Van Roy) S4}
% \author{ }
% \date{October 2014}
% 
% \begin{document}
% % partie 1 (0 à 15 min) Aurélien Pignolet)
% \maketitle

\section{Logique des prédicats}
\subsection{Détails techniques}
\paragraph{}
La première question que l'on se pose est: Comment faire des preuves en logique des prédicats ? Il faut noter que les quantificateurs $\forall$ et $\exists$ rendent le raisonnement plus subtile. Il faudra des règles pour pouvoir raisonner sur ces quantificateurs.
\subsection{Sémantique}
\paragraph{}
La sémantique en logique des prédicats est très proche de celle en logique propositionnelle. Cependant, l'interprétation, I, est plus précise qu'en logique propositionnelle mais est également plus compliquée à utiliser à cause des variables et des symboles de fonction. cette interprétation peut être décrite comme une paire: $I = pair(D_{I}, val_{I}) $ avec D le domaine de discours I et la fonction val qui est l'interprétation de tous les symboles. $D_{I} \ne \emptyset $ $ \forall s \in S$ avec s soit un symbole de prédicat soit une fonction. $val_{I}(S) = P_{I}$ une fonction $P_{I}:D_{I}^{n} \rightarrow (True,False)$ avec S symbole d'une fonction. Il existe aussi une vrai fonction $f_{I}:D_{I}^{n} \rightarrow D_{I}$ avec n le nombre d'arguments tel que $val_{I}(S) = f_{I} $. Cela implique que dans le domaine de discours chaque fonction correspond à une vrai fonction et chaque prédicat correspond à un vrai prédicat. 
\paragraph{}
Si on rajoute une variable, x par exemple, l'expression devient: $ (var, x) \rightarrow val_{I}(x) = x_{I} \in D_{I}$. L'interprétation d'une variable, $x_{i}$ est un élément (n'importe lequel en fonction de l'interprétation) de $D_{I}$. La fonction $VAL_{I}$ est la même fonction mais pour les formules et pas uniquement pour les symboles comme avant. Néanmoins on ne doit pas redéfinir $VAL_{I}$ car elle existe à partir du moment ou $val_{I} et D_{I}$ existent. Définition: $VAL_{I}: TERM \cup  PRED \rightarrow D_{I} \cup (True, False) $ avec TERM l'ensemble des termes et PRED l'ensemble des
prédicats (toutes les formules en logique des prédicats). Les termes peuvent être définis comme $ t \rightarrow VAL_{I}(t)$ et les prédicats comme $ P \rightarrow VAL_{I}(P)$ avec P une formule.Il y à une relation entre 

$val_{I} $ et $ VAL_{I}: val_{I}((P(T_{1},...,T_{n})) = P_{I}(VAL_{I}(t_{1}),...,VAL_{I}(t_{n}))$ avec $ P_{I} = VAL_{I}(P)$ 
La première partie de l'égalité est un prédicat avec des arguments tandis que la seconde est formée de plusieurs prédicats avec un argument. 
\paragraph{}
Définissons d'autres formules. $VAL_{I}(P \wedge Q) $ est true si $VAL_{I}(P) = True$ et $VAL_{I}(Q) = True$. false sinon. Donc il faut $VAL_{I}(\forall x.P) = True$ avec P une formule pouvant dépendre de x (il faut remplacer toutes les occurrences de x dans P et la formule doit restée vraie. Si pour chaque d $\in D_{I}$ $ I' = I \cup (x \leftarrow d)$ et $val_{I}(x) = d$ alors l'interprétation de x doit être: $VAL_{I}'(P) = True$.
Pour $VAL_{I}(\exists x.P) = True$ le raisonnement est identique en remplacant $\forall$ par $\exists$.
\subsection{Différence avec la logique des propositions}
Les trois différences entre les deux logiques sont: les variables, les quantificateurs et les symboles de fonction (moins important). Imaginons un modèle $B: 
\left\{
  \begin{array}{rcr}
    P_{1},...P_{n}
  \end{array}
\right\}
$
Si nous utilisons une interprétation I pour B cela donne:
$\forall P_{I} \in B VAL_{I}(P_{I}) = True$ qui est très générale car $P_{I}$ peut avoir des variables, des quantificateurs, etc...

% partie 1 (15 à 30 min) Thomas)

\subsection{Preuves avec règles}

\paragraph{}
    Il est possible de faire des preuves avec des règles de preuve. Une preuve est une simple manipulation de symboles qui sont des règles. Mais il faut justifier ces règles pour obtenir des résultats vrais. Elles sont justifiées en raisonnant sur les interprétations pour vérifier si elles sont correctes ou non. Tout cela est la sémantique. C'est la base pour pouvoir faire des preuves.
 \paragraph{}
    
Beaucoup de choses restent les mêmes que dans la logique des propositions. La preuve est toujours un objet mathématique, une séquence avec des formules, des justifications, une application des règles. Elle commence avec des prémisses et finit par une conclusion.

\paragraph{}
Il est possible de faire des preuves manuelles, mais aussi des preuves automatisées. C'est une généralisation de l'approche de la logique des propositions. 
\paragraph{}

Il y a toujours :

\begin{itemize}

  \item Une règle de résolution pour les preuves automatisées, mais celle-ci est plus générale. Elle va utiliser un concept appelé "unification". Ce nouveau concept est introduit à cause des variables. En effet, celle-ci peuvent être différentes, il faut donc trouver un nouveau moyen de les fusionner.
  \item Une forme normale, mais elle est plus compliquée à cause des quantificateurs et des symboles de fonctions. Mais il est encore possible de l'obtenir.
  \item un algorithme avec ses propriétés. Mais il est moins fort/complet que l'algorithme développé pour la logique des propositions. Il ne sera plus décidable mais seulement semi-décidable. C'est-à-dire que parfois il tournera en boucle. Cela est dû aux variables et aux quantificateurs. La logique des prédicats est beaucoup plus riche que la logique des propositions, mais en contrepartie l'algorithme arrive à prouver moins de choses. Cependant, l'aglorithme sera toujours adéquat mais pas forcément complet. Il ne sera pas toujours possible de trouver une preuve, même quand elle existe parce qu'elle sera trop compliquée.

\end{itemize}

\paragraph{}
Qu'est-ce qu'il est possible de faire avec ce genre d'algorithme qui est moins fort?

\paragraph{}
Il y a deux possibilités : 
\paragraph{}

1) Faire un assistant de preuve

\paragraph{}
    C'est un outil qui aide les gens à faire des preuves formelles. Deux exemples d'assistants de preuves sont Coq et Isabelle. C'est un outil très sophistiqué mais qui a permis de prouver des choses de manière totalement formelle, alors qu'avant des preuves prenaient des dizaines voire des centaines de pages de preuves mathématiques. Mais cet assistant ne fait pas tout parce que l'algorithme est moins bon. Cependant il aide beaucoup. C'est à l'être humain de lui donner des coups de pouce sous forme de lemmes, hypothèses, chemins, stratégies ... Ensuite l'algorithme s'occupe de la manipulation des symboles.
\paragraph{}
    
    Un exemple très célèbre est le théorème de la coloration d'une carte. La question est : Est-il toujours possible de colorier chaque pays avec une couleur, de façon à ce que deux pays limitrophes n'aient pas la même couleur et en utilisant un certain nombre de couleurs différentes? Ce n'est pas évident à prouver et ça a demandé beaucoup de travail aux mathématiciens. Mais récemment, Georges Gonthier (un informaticien) a réussi à formuler ce problème avec l'assistant de preuves. Ce fût un tour de force. Désormais, il existe une preuve complètement formalisée, sans erreur pour ce théorème.

\paragraph{}
2) L'utiliser dans les langages de programmation
\paragraph{}

    L'algorithme peut être considéré comme le moteur d'un programme. C'est ce qu'on appelle maintenant la programmation logique. Elle consiste à utiliser la logique dans un programme. Le langage le plus célèbre qui a suivi cette approche est Prolog. Ce fût un énorme succès car les gens ne croyaient pas que c'était possible de faire un programme en logique qui pouvait tourner. Cela a donné naissance à la programmation par contraintes (une contrainte est une relation logique). Cette discipline est très utile pour les optimisations, par exemple dans le cas du "voyageur de commerce".

\paragraph{}
La logique des prédicats n'est donc pas quelque chose de seulement théorique, destiné uniquement aux mathématiciens. Les gens ont vraiment essayé avec succès d'utiliser la logique dans l'exécution des programmes.
\paragraph{}

%partie 3 de 30 à 45 min (Guillaume)
\section{Preuves en logique des prédicats}

On va généraliser l'approche de la logique propositionnelle car le langage des prédicats est beaucoup plus riche.  Il ajoute entre autres :

\begin{itemize}
    \item Des variables
    \item Des constantes
    \item Des fonctions (détaillé plus tard)
    \item Des prédicats
    \item Des quantificateurs
\end{itemize}

Les preuves en logique des prédicats ressemblent très fort aux preuves en logique propositionnelles. Il y a encore des prémisses, des formules avec leurs justificatifs et une conclusion. On peut aussi utiliser des preuves indirectes et des preuves conditionnelles. Cela reste un objet formel.

\begin{center}
\fbox{
$
\begin{array}{l l l}
  1. & \fbox{\ldots} &  Premisses \\
  2. & \fbox{Formule, regle} &  Justification \\
  \ldots & \ldots &  \ldots \\
  n. & \fbox{Conclusion} &  Justification \\
\end{array}
$
}
\end{center}
\subsection{Exemple}

La méthode pour passer de $\forall x \cdot P(x) \wedge Q(x)$ (prémisse) à $\forall x \cdot P(x)\wedge(\forall x \cdot Q(x))$ (conclusion) est la suivante:
\begin{center}
$
\begin{array}{l l}
  1. & $Enlever les quantificateurs pour avoir des variables libres$ \\
  2. & $Raisonner sur l'intérieur$\\
  3. & $Remettre les quantificateurs$\\
\end{array}
$
\end{center}
Les étapes difficiles à réaliser correctement sont les étapes $1$ et $3$. Voici la preuve en "français" :

En retirant les quantificateurs des prémisses, cela donne :
"Alors, $P(x) \wedge Q(x)$ est vrai pour tout $x$ donc $P(x)$ est vrai pour tout $x$". 
De là on peut remettre les quantificateurs pour obtenir $\forall x \cdot P(x)$. De façon similaire, on obtient $\forall x \cdot Q(x)$. Et on conclut en remettant les quantificateurs: $\forall x \cdot P(x) \wedge \forall x \cdot Q(x)$, en utilisant la conjonction.

En preuve formelle cela donne :

\begin{center}
\fbox{
$
\begin{array}{l l l}
  1. &  \forall x \cdot P(x) \wedge Q(x) &  $Prémisses$ \\
  2. & P(x) \wedge Q(x) &  $Elimination de $\forall \\
  3. & P(x) &  $Simplification$ \\
  4. & \forall x \cdot  P(x) &  $Introduction de $ \forall \\
  5. & Q(x) &  $Simplification $ \forall \\
  6. & \forall x \cdot Q(x) &  $Introduction de $\forall \\
  7. & \forall x \cdot P(x) \wedge \forall x \cdot Q(x) &  $Conjonction$ \\
\end{array}
$
}
\end{center}

On a donc utilisé 4 règles en plus par rapport aux preuves formelles en logique propositionnelle (les règles de la logique propositionnelle restent valables en logique des prédicats):
\begin{itemize}
\item Elimination de $\forall$
\item Introduction de $\forall$
\item Elimination de $\exists$
\item Introduction de $\exists$
\end{itemize}

Certaines de ces règles sont simples d'utilisation, d'autres sont plus difficiles. Il est également possible d'utiliser d'autres règles (certaines plus générales que d'autres\footnote{Voir "Inference logic" ou "Predicate logic"}).

\begin{framed}
\textbf{Note} :\\

Il est possible d'utiliser les quantificateurs  dans les formules mathématiques. Typiquement, on ne les note pas, car ils sont présents de manière implicite. Par exemple :
\begin{itemize}
\item $\forall x \cdot \sin(2x) = 2 \cdot \sin(x) \cdot \cos(x)$ 
\item $\forall x \cdot x + x = 2x$
\item $\exists x \cdot \sin(x) + \cos(x) = 0,5$
\item $\exists x \cdot x + 5 = 9$
\end{itemize} 

On peut remarquer que pour les deux premiers cas, $x$ est une véritable variable, on peut donc ajouter un quantificateur universel $\forall$. 
Pour les deux cas suivants, on remarque que $x$ est une inconnue, car il y a une équation à résoudre et une solution à trouver, on peut donc ajouter un quantificateur existentiel $\exists$.\\

Dans certains cas, les quantificateurs existentiels et universels sont utilisés au sein de la même formule mathématique.
\begin{itemize}
\item[] $\forall a \cdot \forall b \cdot \forall c \cdot \exists x \cdot ax^{2}+bx+c = 0$
\end{itemize}
Dans l'exemple ci-dessus, nous avons 4 variables : $a,b,c,x$. Les 3 premières sont des véritables variables, on peut les affecter à n'importe quelle valeur, tandis que la dernière est une inconnue, c'est la solution à trouver. Il faut donc trouver $x$ pour toutes les valeurs possibles de $a,b,c$.
\end{framed}

\section{Règles en logique des prédicats}
En logique des prédicats, pour trouver une preuve, on va faire des manipulations de formules.

\subsection{La substitution}
Une manipulation fréquente en logique des prédicats est la \textbf{substitution}. Elle consiste à prendre une formule et remplacer une partie par une autre.\\

Si on a une formule $p[x/t]$ (où $p$ veut dire "toutes les formules"). On va remplacer toutes les occurrences libres de $x$ par $t$.\\
On peut aussi écrire cela de la manière suivante :\\

$p[x/t]$ possède deux portées :
\begin{itemize}
\item[] La première se note $p[x]$ et correspond à une partie de la règle.
\item[] La seconde se note $p[t]$ et correspond à l'autre partie de la règle.\\
\end{itemize}

Exemple :
\begin{center}
\begin{tabular}{|l |l |>{\raggedright}m{6cm}|}
\hline
1. &$P(x) \rightarrow \forall y \cdot (P(x) \wedge R(y))$&$[x/y]$ veut dire qu'on va remplacer toutes les occurences libres de $x$ par $y$.\tabularnewline
\hline
2. &$P(y) \rightarrow \forall y \cdot (P(y) \wedge R(y))$&En remplaçant $x$ par $y$, on a changé le sens de la formule car avant, $x$ n'était pas dans la portée du quantificateur alors que maintenant il l'est. Ce changement de sens s'appelle une \textbf{capture de variable}, car la variable $y$ est capturée par le quantificateur. Pour résoudre ce problème, on va effectuer un \textbf{renommage}.\tabularnewline
\hline
3. &$P(y) \rightarrow \forall z \cdot (P(y) \wedge R(z))$&Résultat après renommage (pour éviter la capture de variable).\tabularnewline
\hline
\end{tabular}
\end{center}

% \end{document}
