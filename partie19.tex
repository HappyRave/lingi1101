Reprenons l'exemple de la section 31.4, si l'on s'intéresse à la personne A, on sait que s' il cherche du travail c'est fort probable que ce soit via B qu'il en trouve (B est une connaissance de A).Socialement, on sait que si deux amis ont un ami en commun, il y a des chances qu'ils deviennent ami aussi ou tout du moins connaissance (propriété de fermeture triadique forte).On peut donc en déduire que chaque pont local tel que celui entre A et B dans la figure 8 sera un lien faible (conséquence de la propriété de fermeture triadique forte).En effet si le lien entre A et B était fort, la propriété de fermeture triadique forte nous dirais que d'autres liens se formeraient. Par exemple entre E et B et entre A et F ce qui aurait pour conséquence que le lien A-B ne serait plus un pont.B fourni donc de nouvelles informations à A via le pont local.
\newline
\subsection{Force d'un lien dans un grand réseau}
Pour caractériser la force d'un lien dans un grand réseau, il va falloir généraliser les concepts relatifs aux liens forts et faibles vus précédemment.
La force d'un lien sera caractérisée en y attribuant une quantité numérique, on va utiliser la notion de chevauchement de voisinage (neighbourhood overlapping) pour ça.
Chevauchement de voisinage = (Nombre de noeuds voisins de A et B)/(Nombre de noeuds voisins de A ou B)
La quantité numérique augmentera en fonction de ce chevauchement. Plus cette valeur de chevauchement de voisinage tendra vers 0, plus ce lien deviendra un pont local.
\newline
\subsection{Force des liens en pratique dans un réseau téléphonique}
Une étude portée sur les conversations cellulaires nous prouve que plus les liens entre des invidus sont fort, plus ces personnes passeront du temps au téléphone à communiquer. Effectivement, on remarque que la durée augmente au fur et à mesure que le chevauchement de voisinage augmente.
\newline
\subsection{Force des liens en pratique sur facebook}
Facebook est un réseau social développé sur internet permettant à des indivdus de communiquer avec des personnes de leur entourage.
Les liens entre individus sur Facebook sont nommés les liens d'amitié.
Facebook est ainsi notamment un bon exemple où la force des liens est utilisée au maximum.
D'après une étude réalisée sur une période d'un mois, on trouve trois catégories de liens sur Facebook, caractérisant soit:
- une communication réciproque;
- une communication orientée;
- une relation maintenue.

On peut faire une comparaison avec la force d'un lien, le nombre de personnes ayant un lien d'amitié avec quelqu'un augmente en fonction de la taille du voisinage. 
On remarque également que la propagation d'informations est plus rapide avec la 3 catégorie de lien.
\newline
\subsection{Twitter}
Twitter est un microblog qui permet de partager de petits messages (140 catactères au maximum). Il est basé sur une relation de Followers à Followees.
Force du lien :
- Faible si on est follower
- Forte si on envoie un message ciblé
Si on "suit" plus de personne, on a plus d'amis (jusqu'à un certain point, c'est une fonction logarithmique). Effectivement, cela s'explique par le fait que pour entretenir des liens forts, ceci demande un effort conditionnel. Il y a une limite physique aux nombres de liens forts que l'on peut posséder. A contrario, il ne demande pas de temps ou d'effort pour 'suivre' quelqu'un. Il y a moins de contraintes, c'est un engagement passif et c'est ce qui est proné sur Twitter.