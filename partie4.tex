
\chapter{Deux règles plus sophistiquées}

\section{Théorème de déduction}

\begin{itemize}
\item  Pour prouver s $\Rightarrow$ t
\item  On suppose s vrai
\item  On déduit t
\item  Ensuite on évacue l'hypothèse
\end{itemize}


\textit{Notation: p $\vdash$ t (on peut prouver t à partir de p) }

\subsection{Prémisse:}

\begin{equation}
\frac{p........................r, s \vdash t} 
{p........................r, s \vdash (s \Rightarrow t)}
\end{equation}

\subsection{Conclusion:}

Déduire une implication

\section{Preuve par contradition (ou preuve indirecte)}

On prend un hypothèse, et on peut prouver qu'elle est vraie ou fausse, d'où l'hypothèse n'est pas bonne.

\subsection{Prémisse:} 
on suppose que p...q n'a pas de problème

\begin{equation}
\begin{split}
p........................q, r, s \vdash s \\
\frac{p........................ q,r, s \vdash \lnot s}
{p........................ q \vdash \lnot r}
\end{split}
\end{equation}

\subsection{Conclusion:}

si p...q n'a pas de problème, on se focalise alors sur r
 

\chapter{Exemples de Preuves}

Une preuve est une séquence de pas où chaque pas est une application de règles d'inférences et de lois logiques
Il faut justifier à chaque étape le nom de la règle / loi, et indenter les éléments de la preuve en preuve conditionnelle indirecte.

\section{Prémisse:} 

p $\land$ q $\lor$ r

\section{Conclusion:}

$\lnot$ p $\Rightarrow$ r

\section{Exemple sans preuve conditionnelle}

\begin{enumerate}
\item   (p $\land$ q) $\lor$ r  \textit{Prémisse}
\item   r $\lor$ (p $\land$ q)  $\lor$ r \textit{Commutativité en 1}
\item   (r $\lor$ p) $\land$ (r$\lor$q) \textit{Associativité en 2}
\item   (r $\lor$ p) \textit{Simplification en 3}
\item   (p $\lor$ r) \textit{Commutativité en 4}
\item   $\lnot$$\lnot$ p $\lor$ r \textit{Loi de la négation en 5}
\item   $\lnot$ p $\Rightarrow$ r \textit{Implication en 6}
\end{enumerate}

\section{Exemple avec preuve conditionnelle}

\begin{enumerate}
\item (p $\land$ q) $\lor$ r  \textit{Prémisse}
\item $\lnot$ $\lnot$(p $\land$ q) $\lor$ r \textit{Double négation en 1}
\item $\lnot$ ( $\lnot$ p $\land$ $\lnot$ q) \textit{Loi De Morgan en 2}
\item $\lnot$ p $\lor$ $\lnot$ q $\Rightarrow$ r \textit{Implication en 3}

\begin{enumerate}
 \item  $\lnot$ p \textit{Hypothèse}
 \item  $\lnot$ p $\lor$ $\lnot$ p \textit{ Addition sur 5}
 \item  r \textit{ Modus Ponens sur 4 et 6}
\end{enumerate}

\item  $\lnot$ p $\Rightarrow$ r \textit{Evacuation de l'hypothèse}
\end{enumerate}

\section{Exemple de Preuve par contradiction}

\begin{enumerate}
\item  (p $\land$ q) $\lor$ r  \textit{Prémisse}
\item  (p  $\lor$ r) $\land$ (q $\lor$ r) \textit{Distributivité sur 1}
\item  (p $\lor$ r)  \textit{Simplification en 2}

\begin{enumerate}
 \item $\lnot$ ( $\lnot$ p $\Rightarrow$ r)  \textit{Hypothèse}
 \item $\lnot$ ( $\lnot$ $\lnot$ p $\lor$ r)\textit{Implication en 4}
 \item $\lnot$ (p $\lor$ r) \textit{ Négation en 5}
\end{enumerate}

\item $\lnot$ $\lnot$ ($\lnot$ p $\Rightarrow$ r)  \textit{ Preuve par contradiction}
\item $\lnot$ p $\Rightarrow$ r \textit{Négation en 7}
\end{enumerate}


\chapter{Quelques concepts}
\section{Principe de dualité }

\subsection{Dans les formules sans $\rightarrow$ :}

\begin{equation}
1 \leftrightarrow T \hspace{5cm} true \leftrightarrow false 
\end{equation}
\begin{align*}
\models \lnot ( p \land q)  \Leftrightarrow \lnot p \lor \lnot q \\
\models \lnot ( p \lor q)  \Leftrightarrow \lnot p \land \lnot q 
\end{align*}

\subsection{Formule quelconque:}

\begin{equation}
	1 \leftrightarrow T \hspace{5cm} true \leftrightarrow false 
\end{equation}

 Justification en raisonnant sur les modèles:
 \begin{align*}
	 p1,...,pn \models ssi \models (p1,...,pn \land \lnot q) \leftrightarrow false \\
	 ssi \models ( \lnot p1 \lor ... \lor pn) \leftrightarrow true
 \end{align*}

\section{Forme Normale}

\begin{itemize}
  \item Conjonctive: $( p \lor q ) \land ( q \lor a ) \land ( s \lor r )$  
  \item Disjonctive
\end{itemize}

\subsection{terminologie}

\begin{itemize}
	\item Littéral :$ P \lor \lnot P \approxeq L$
	\item Clause : $\lor L{i} = ( L{1} \lor L{2} \lor L{3} ... \lor L{i} )$
\end{itemize}

\subsection{Forme normale conjonctive FNC}

\section{Algorithme de normalisation}

\begin{enumerate}
\item Eliminer les $\rightarrow$ et $\leftrightarrow$
\item Déplacer les négations vers l'intérieur (dans les propositions premières) De Morgan
\item Déplacer les disjonctions ($\lor$) vers l'intérieur
\item Simplifier $(P \lor \lnot P)$
\end{enumerate}

\subsection{Exemple}

\begin{align*}
& (p \rightarrow (Q \rightarrow R)) \rightarrow ((P \land S) \rightarrow R) \\
& \lnot ( ... ) \lor ( ... ) \\
& \lnot (\lnot P \lor (\lnot Q \lor R)) \lor (\lnot (P \land S) \lor R) \\
& ( \lnot \lnot P \land \lnot (\lnot Q \lor R)) \lor ((\lnot P \lor \lnot S) \lor R) \\
& (P \land (Q \land \lnot R)) \lor ( \lnot P \lor \lnot S \lor R) \\
& (P \lor \lnot P \lor \lnot S \lor R) \land ( Q \lor \lnot P \lor \lnot S \lor R) \land (\lnot R \lor \lnot P \lor \lnot S \lor R) \\
& (Q \lor \lnot P \lor \lnot S \lor R) 
\end{align*}

